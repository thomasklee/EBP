% EBP course: session 10
% Thomas Klee
% 10 April 2019

% Hide last 4 slides when uploading PDF to Moodle. 
% Lecture from a second PDF which shows those slides.

% Preamble
\documentclass{beamer}
\usetheme{Singapore}
\usefonttheme[onlysmall]{structurebold}
\setbeamerfont{title}{shape=\itshape,family=\rmfamily}
\usepackage{graphicx}
\usepackage[english]{babel}
\usepackage[utf8x]{inputenc}
\usepackage{amsfonts, amsmath, amsthm, amssymb} % for math fonts, symbols and environments
\usepackage{xcolor}
\usepackage{booktabs}
\usepackage{ctable} % for command-driven tables
\usepackage{wasysym}
\usepackage[natbibapa]{apacite}
\beamertemplatenavigationsymbolsempty % uncomment to add slide navigation symbols to each slide
\setbeamertemplate{footline}[frame number]
\usepackage{appendixnumberbeamer}  % to suppress page numbers on extra slides

% activate following line for custom appearance
% \usepackage{beamerthemesplit} 

% information for title slide
\title{Clinical Guidelines, Audits, \\ Implementation \& Barriers}
\subtitle{}
\author{Evidence-Based Practice in Speech-Language Therapy \\ (SHSC 2033)}
\institute{Session 10}
\date{Thomas Klee \& Elizabeth Barrett}
\titlegraphic{\includegraphics[width=6cm]{images/logo_CE_C.jpg}} % HKU logo

\begin{document}

% create title slide with information above
\begin{frame}
	\titlepage
\end{frame}

% 
\begin{frame}{Outline}
	\begin{enumerate}
	\item Clinical guidelines and evidence maps
	\item Clinical audit	
	\item Group discussion: implementing EBP and possible barriers
	\end {enumerate}
\end{frame}

\section{Clinical Guidelines}

%
\begin{frame}
\center{\Huge{\textcolor{darkgray}{Clinical Guidelines}}}
\end{frame}

% 
\begin{frame}{Clinical practice guidelines}
	\begin{quote}
	``CPGs may contribute to solving the limited resources/time dilemma because they contain secondary research that enables practitioners to access literature easily and quickly that has already been identified, summarized, and critiqued." \footnote{\tiny{\citet[p. 290]{Hargrove2008}}}
	\end{quote}
\end{frame}

% 
\begin{frame}{Clinical practice guidelines}
	\begin{quote}
	``This \emph{prereviewed} literature allows SLPs to expend fewer resources and less time on the \emph{evidence} part of EBP. In addition, CPGs are likely to reduce error and bias\dots because CPGs are based on organized systematic literature selection and predetermined analysis strategies." \footnote{\tiny{\citet[p. 290]{Hargrove2008}}}
	\end{quote}
\end{frame}

% 
\begin{frame}{Three types of CPGs \footnote{\tiny{\citet[p. 290]{Hargrove2008}}}}
	\begin{enumerate}
	\item Traditional CPGs
	\item Systematic reviews
	\item Evidence-based CPGs
	\end{enumerate}
\end{frame}

% 
\begin{frame}{Evidence-based CPGs \footnote{\tiny{\citet[Table 1]{Hargrove2008}}}}
	\begin{itemize}
	\item Based on a comprehensive, methodical literature review
	\item Based on a consensus of a panel of experts
	\item Can include expert opinions
	\item Identifies evidence that support recommendations
	\item Evaluates the quality of the literature used to support recommendations
	\item Expertise of the expert or group of experts is disclosed
	\item Can include case studies, retrospective, nonrandomized research designs
	\end{itemize}
\end{frame}

% 
\begin{frame}{CPGs}
	\begin{itemize}
	\item ``Although we were unable to identify a system of grading levels of evidence that included EBCPGs, one could infer that they generally would fall between TCPGs and SRs." \footnote{\tiny{\citet[p. 290]{Hargrove2008}}}
	\end{itemize}
\end{frame}

% 
\begin{frame}{Examples of CPGs}
	\begin{itemize}
	\item Royal College of Speech and Language Therapists' document \citep{Taylor-Goh2005} summarising 12 clinical guidelines for children and adults \emph{``regarding clinical management that are based on the current evidence, where available " } \footnote{\tiny{Search for "clinical practice guidelines" at \url{{https://www.rcslt.org}}}}
	\item SIGN's 2016 guidelines for ASD (\#145) \footnote{\tiny{https://www.sign.ac.uk/our-guidelines.html}}
	\item Hargrove et al.'s (2008, Appendix A) for a list of CPGs relevant to children with communication disorders
	\end{itemize}
\end{frame}

%
\begin{frame}{Two cautions about CPGs}
	\begin{itemize}
	\item Note date of publication. Guidelines must be kept current if they are to have value.
	\item Treat CPGs as you would any other piece of evidence: \alert{critically appraise them}.
	\end{itemize}
\end{frame}

% 
\begin{frame}{Critical appraisal}
	\begin{itemize}
	\item Grading of Recommendations Assessment, Development and Evaluation (GRADE) \footnote{\tiny{\citet[pp. 281ff]{Guyatt2008d}}}
	\item Guide to Evaluating Clinical Practice Guidelines \footnote{\tiny{\citet{Hargrove2008}}}
		\begin{itemize}
		\item[-] 19 item checklist and instructions
		\item[-] Excellent inter-rater reliability (although small sample)
		\item[-] Item 3e: \emph{``Were each of the recommendations of the CPG linked to evidence?"} (no instructions provided on this one)
		\end{itemize}
	\end{itemize}
\end{frame}

% 
\begin{frame}{ASHA's evidence maps}
	\begin{itemize}
	\item Summaries of clinical research related to assessment and intervention in communication disorders
	\item See \url{http://on.asha.org/evidence-maps}
	\item Also, see \url{http://www.asha.org/Research/EBP/EBSRs}
	\item For a clinician's POV, see \citet{VanDyke2018}
	\end{itemize}
\end{frame}

% 
\begin{frame}{Remember: \alert{Do not cherry pick!}}
	\begin{itemize}
	\item Choosing what you want to believe by ignoring some studies or not critically appraising them.
	\item Example: Not having your child vaccinated because \emph{``MMR vaccine causes autism."} 
	\item Searching for, and critically appraising the evidence, should help you avoid this pitfall. 
	\end{itemize}
\end{frame}

% 
\begin{frame}{E$^3$BP}
	\begin{quote}
	``\dots the conscientious, explicit, and judicious integration of  best available\\
		\begin{enumerate}
		\item external evidence from systematic research,
		\item evidence internal to clinical practice, and
		\item evidence concerning the preferences of a fully informed patient." \footnote{\tiny{\citet[p. 2]{Dollaghan2007a}}}
		\end{enumerate}
	\end{quote}
\end{frame}

% 
\begin{frame}{No one is an island.\footnote{\tiny{Paraphrase of John Donne (1624)}}}
	\begin{itemize}
	\item You now have a framework for EBP. 
	% You've probably had more formal training in EBP than any other ST in Hong Kong!
	\item What are the challenges to actually using it to inform your clinical decision making?
	\item One challenge is that you won't be working in isolation. Other people will be involved.
	\item You will be part of a \alert{team}.
	\item You will also be part of a larger institutional structure.
	\end{itemize}
\end{frame}

\section{Clinical Audit}

%
\begin{frame}
\center{\Huge{\textcolor{darkgray}{Clinical Audit}}}
\end{frame}

% 
\begin{frame}{Clinical audit \footnote{\tiny{\url{http://www.hra-decisiontools.org.uk/research/glossary.html}}}}
	\begin{itemize}
	\item ``\dots a quality improvement process that seeks to improve patient care and outcomes through systematic review of care against explicit criteria and the implementation of change.
	\item Aspects of the structure, processes, and outcomes of care are selected and systematically evaluated against explicit criteria. 
	\item Where indicated, changes are implemented at an individual, team, or service level and further monitoring is used to confirm improvement in healthcare delivery."
	\end{itemize}
\end{frame}

% 
\begin{frame}{Clinical audit \footnote{\tiny{\url{http://www.uhbristol.nhs.uk/files/nhs-ubht/1\%20What\%20is\%20Clinical\%20Audit\%20v3.pdf}}}}
	\begin{itemize}
	\item A strategic management tool used as part of a broader quality improvement programme
	\item A specific activity that measures clinical care against explicit audit criteria (standards) as part of a quality improvement cycle 
	\item Necessary to know exactly what you do already and how that works, before you try to change it.
	\item Asks the questions ``are we following best practice?" and ``what is happening to patients as a result?" 
	\end{itemize}
\end{frame}

% 
\begin{frame}{Audit cycle \footnote{\tiny{\url{http://www.uhbristol.nhs.uk/files/nhs-ubht/1\%20What\%20is\%20Clinical\%20Audit\%20v3.pdf}}}}
	\begin{enumerate}
	\item Choose topic
	\item Review standards of best practice (audit criteria)
	\item Collect data of current practice
	\item Compare data with standards
	\item Feed back results
	\item Discuss possible changes
	\item Implement changes if needed
	\end{enumerate}
\end{frame}

% 
\begin{frame}{Evaluating clinical practice}
	\begin{itemize}
	\item EBP informs the setting of standards and the change in practice. Without linking the \alert{evidence base} to the \alert{clinical audit cycle}, we continue to practice based on our beliefs rather than on sound information.
	\item What happens if an organisation has no audit cycle?
	\end{itemize}
\end{frame}

% 
\begin{frame}{Questions to consider}
	\begin{itemize}
	\item How do you know if what \alert{you} do works? % Examine single-subject data; use CAPE
	\item How do you monitor quality in \alert{your} clinical practice? % Clinical audit
	\item How do you bring about a \alert{change} in practice?
	\end{itemize}
\end{frame}

\section{Group Discussion}

% 
\begin{frame}{Group discussion: implementation \& barriers}
	\begin{itemize}
	\item Break up into your assigned groups.
	\item Discuss
		\begin{enumerate}
		\item What are some of the skills you already have as a Speech Therapy student?
		\item What barriers do you think you might encounter trying to implement EBP in your clinical work as a student?
		\end{enumerate}
	\item Make notes in order to communicate your findings at the end.
	\end{itemize}
\end{frame}

% Hide the next 2 slides when uploading PDF to Moodle.
% Reveal them after the group discussion.

% 
\begin{frame}{Implementing EBP}
	\begin{itemize}
	\item Some skills you already have
		\begin{itemize}
		\item[-] Academic knowledge
		\item[-] Clinical experience
		\item[-] Problem-solving skills
		\item[-] Strategies for finding high quality evidence
		\item[-] Some statistical knowledge
		\item[-] Critical appraisal skills
		\item[-] Communication skills
		\end{itemize}
	\item All involve \alert{life-long learning}.
	\end{itemize}
\end{frame}

% 
\begin{frame}{Barriers to implementing EBP}
	\begin{enumerate}
	\item Complacency; inertia 
	\item Availability of evidence
	\item Knowledge-practice gap
	\item[]
	\item Tradition of trial-and-error problem-solving
	\item Can feel threatening to colleagues
	\item Professional autonomy 
	\item Lack of colleagues' support 
	\item[]
	\item Current work culture
	\item Management inertia and poor leadership
	\item The cost of change
	\end{enumerate}
\end{frame}

\begin{frame}[allowframebreaks] %[shrink=15] % to reduce font size of references
	\begin{center}
	\frametitle{References}
	\bibliographystyle{apacite}
	\small\bibliography{/Users/thomasklee/Documents/Bibtex/library}
	\end{center}
\end{frame}

\end{document}%%