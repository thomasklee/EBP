% Handout: writing tips for Moodle
% Author: Thomas Klee
% 25 Feb 2017

% Preamble --------------------------------------------------------

\documentclass[a4paper]{article}
\usepackage[]{fullpage}
\usepackage[latin1]{inputenc}
\usepackage{multicol}
\usepackage[colorlinks = true, linkcolor = blue, urlcolor = blue, citecolor = black]{hyperref} 
\usepackage[natbibapa]{apacite}
\usepackage{amsmath}
\usepackage{amsfonts}
\usepackage{amssymb}
\usepackage{textcomp}
\usepackage{booktabs}
\usepackage{caption}

% Course title etc ---------------------------------------------------

\title{Writing Tips}
\author{Thomas Klee  \\ University of Hong Kong}
\date{Evidence-Based Practice in Speech-Language Therapy (SHSC 2033)}

% Document ---------------------------------------------------------------

\begin{document}

\maketitle

\begin{enumerate}

\item Place yourself in the reader's shoes by asking whether your paper conveys the essential information in a structured and clear manner. Remember that your reader may not have read all the things you've read, so make sure what you've written is embedded in an appropriate context.

\item Use headings (and subheadings) to delineate major sections of your paper. They provide a road map for the reader.

\item Cite the article(s) you are referring to using in the way recommended in the APA Publication Manual. If tables, figures, or appendices are part of your paper, make sure you?ve referred to those in the text. 

\item Get to the point when you write. Try to convey the essence of the rationale, methods, results and possible implications of the study succinctly.

\item Check grammar and spelling. 

\item Don't start sentences with numerals. For example, write \emph{Six participants were tested.}, not \emph{6 participants were tested.} 

\item Report summary scores such as Means and SDs whenever you report effect sizes and p-values. If the authors don't report an effect size, try calculating it yourself based on the summary scores in the paper using the effect size calculator at \url{http://www.polyu.edu.hk/mm/effectsizefaqs/calculator/calculator.html}.

\item Don't only think about what is written in the paper but also think about what you would have preferred to see in the paper when doing your critical appraisal.

\item You can use information from other papers to support your claims. Refer to these papers in your text.

\item Check the APA Publication Manual for how to cite studies (e.g., when to use \emph{et al.} in a citation and what parts of the reference should be italicized).

\end{enumerate}

\end{document}