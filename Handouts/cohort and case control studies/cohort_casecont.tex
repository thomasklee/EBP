% Handout: cohort and case-control studies
% Author: Thomas Klee
% 4 Feb 2017

% Preamble --------------------------------------------------------

\documentclass[a4paper]{article}
\usepackage[]{fullpage}
\usepackage[latin1]{inputenc}
\usepackage{multicol}
\usepackage[colorlinks = true, linkcolor = blue, urlcolor = blue, citecolor = black]{hyperref} 
\usepackage[natbibapa]{apacite}
\usepackage{amsmath}
\usepackage{amsfonts}
\usepackage{amssymb}
\usepackage{textcomp}
\usepackage{booktabs}
\usepackage{caption}

% Course title etc ---------------------------------------------------

\title{Cohort and Case-Control Studies}
\author{Thomas Klee  \\ University of Hong Kong}
\date{Evidence-Based Practice in Speech-Language Therapy (SHSC 2033)}

% Document ---------------------------------------------------------------

\begin{document}

\maketitle

\begin{center}
\section*{Cohort study}
\end{center}

\subsection*{Study design} 
\begin{itemize}
\item[] Follows a group of people (a cohort) to examine how different events (or risks) affect them; outcomes of people in subgroups of the cohort are compared.
\item[] Usually prospective, but can be retrospective (see \citealp[pp. 118--120]{Ajetunmobi2002}) 
\item[] Longitudinal; observational (non-experimental)
\end{itemize}

\subsection*{Example}
\begin{itemize}
\item[] Smokers and non-smokers were followed over time to determine whether they differ in health outcomes (e.g. lung cancer) \citep{Doll1956}.
\end{itemize}

\subsection*{Outcome measure}
\begin{itemize}
\item[] \textsc{Relative risk} (RR) is ``a ratio comparing the probability of an outcome in those exposed with the probability of that outcome in those unexposed" \citep[p. 107]{Ajetunmobi2002}. See the table on the next page for how this equation is applied to a 2 x 2 outcome table.
\end{itemize}

$$RR = \frac{a / (a + b)}{c / (c + d)}$$ 

\subsection*{Interpretation}
\begin{itemize}
\item[] $RR < 1$	Decreased risk
\item[] $RR = 1$	No difference in risk
\item[] $RR > 1$	Increased risk
\item[] RR can be estimated directly in a cohort study.
\end{itemize}

\begin{center}
\section*{Case-control study}
\end{center}

\subsection*{Study design} 
\begin{itemize}
\item[] Two groups (cases and controls) are examined for differences associated with case status.
\item[] Usually retrospective, observational, non-experimental
\end{itemize}

\subsection*{Examples}
\begin{itemize}
\item[] Children with and with specific language impairment were compared on a variety of factors using a parent questionnaire \citep{Tomblin1997}.
\item[] Late talkers and typically-developing toddlers were compared on a range of maternal, family and child variables \citep{Zubrick2007}.
\end{itemize}

\subsection*{Outcome measure}
\begin{itemize}
\item[] The \textsc{odds ratio} (OR) is defined as ``the ratio of the odds of an event occurring in the experimental group (\emph{cases}) compared with the odds of the same event in the \emph{control} group" \citep[pp. 117--8]{Ajetunmobi2002}. See the table below for how this equation relates to a 2 x 2 table.
\item[] $$OR = \frac{a / b}{c/d} = \frac{a / c}{b / d} = \frac{ad}{bc}$$ 
\end{itemize}

\subsection*{Interpretation}
\begin{itemize}
\item[] $OR < 1$ suggests exposure reduces outcome
\item[] $OR = 1$ suggests exposure has no effect
\item[] $OR > 1$ suggests exposure increases outcome
\item[] OR can be used to estimate relative risk in a case-control study \citep[p. 241]{Bland2000}.
\end{itemize}

\begin{center}
\section*{2 x 2 outcome table}
\end{center}

\begin{center}
\begin{tabular}{l c | c | c}
\toprule
& & Case & Control \\ 
\hline
Predictor & $+$ & a & b \\
\hline
Predictor & $-$ & c & d \\
\bottomrule
\end{tabular}

\end{center}

\bibliographystyle{apacite}
\bibliography{/Users/thomasklee/Documents/Bibtex/library}

\end{document}